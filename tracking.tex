\documentclass[letter]{article}
\usepackage[utf8]{inputenc}
\usepackage[margin=2in]{geometry}
\usepackage{tikz}
\usepackage{ulem}
\usepackage{graphics}
\usepackage{sidecap}
\usepackage{wrapfig}
\usepackage[toc,page]{appendix}
\usepackage[font={small}]{caption}
\usepackage{hyperref}
\usepackage{parskip}
\usepackage{amsmath}
\usepackage{cancel}

\hypersetup{
    colorlinks,%
    citecolor=black,%
    filecolor=black,%
    linkcolor=black,%
    urlcolor=black
}

\def\dashuline{\bgroup 
  \ifdim\ULdepth=\maxdimen  % Set depth based on font, if not set already
	  \settodepth\ULdepth{(j}\advance\ULdepth.4pt\fi
  \markoverwith{\kern.15em
	\vtop{\kern\ULdepth \hrule width .3em}%
	\kern.15em}\ULon}

\newcounter{foot}
\setcounter{foot}{1}

\setlength\parindent{2em}

\DeclareMathOperator*{\argmax}{argmax}
\DeclareMathOperator*{\argmin}{argmin}
\DeclareMathOperator*{\trace}{tr}



\author{Todd Borrowman, Christopher Patton}
\date{\today}
\title{QRAAT}






\begin{document}
\maketitle

\begin{abstract}
Assorted topics on QRAAT. 
\end{abstract}

\tableofcontents
\pagebreak



\section{Position estimation} 

  \subsection{Signal model}
The signal is modeled as a complex vector of length $n$: 
\begin{equation}
  \mathbf{V} = T\mathbf{G}(\theta) + \mathbf{N}
\end{equation}
$T$ is the signal power, $\mathbf{G}(\theta)$ corresponds to the antenna pattern 
with respect to the direction of arrival $\theta$\footnote{Also called the steering 
vector of $\theta$.}, and $\mathbf{N}$ represents 
the Gaussian-distributed signal noise. The probability of $\mathbf{V}$ given the
direction of arrival is used to derive the direction finder:
\begin{equation}
  P(\mathbf{V} | \theta) = \frac{1}{\pi^n \det{R}} \exp{(-\mathbf{V}^HR^{-1}\mathbf{V})}
\end{equation}
where $R$ is the covariance matrix of $\mathbf{V}$. To derive this value, we first 
notice that
$$ E[\mathbf{V}] = E[T\mathbf{G}(\theta) + \mathbf{N}] = E[T\mathbf{G}(\theta)] + E[\mathbf{N}] = 0 + 0 $$
since $\mathbf{V}$ has zero mean if $T$ is zero mean. 
\begin{equation*}
  \begin{aligned}
    R &= E[ (\mathbf{V} - E[\mathbf{V}])(\mathbf{V} - E[\mathbf{V}])^H ] & \text{Def. of covariance.} \\
      &= E[\mathbf{V}\mathbf{V}^H] & E[\mathbf{V}] = 0. \\ 
      &= E[(T\mathbf{G}(\theta) + \mathbf{N})(T\mathbf{G}(\theta)+ \mathbf{N})^H] \\ 
      &= E[TT^H\mathbf{G}(\theta)\mathbf{G}(\theta)^H] + E[\mathbf{N}\mathbf{N}^H] \\
       & \qquad \qquad + E[T\mathbf{G}(\theta)\mathbf{N}^H] + E[T^H\mathbf{G}(\theta)^H\mathbf{N}] \\
      &= E[TT^H]\mathbf{G}(\theta)\mathbf{G}(\theta)^H + E[\mathbf{N}\mathbf{N}^H] + 0 + 0
  \end{aligned}
\end{equation*}
since $\mathbf{G}(\theta)$ is a deterministic function and $T$ and $\mathbf{N}$ are non-coherent. 
We rewrite this equation with the following notation: 
\begin{equation}
    R = \sigma_T^2\mathbf{G}(\theta)\mathbf{G}(\theta)^H + \Sigma
\end{equation}
The signal power covariance is computed as $\sigma_T^2 = t - \trace{(\Sigma)}$, where 
$t$ is equal to the power of $\mathbf{V}$. 



  \subsection{Direction finders}
The direction finder computes the most probable $\theta$ given the signal $\mathbf{V}$:
$$ \argmax_\theta{P( \mathbf{V} | \theta )}$$
The maximum likelihood estimator (MLE) simplifies this computation by taking the natural log 
of the distribution.
\begin{equation}
  \begin{aligned}
    \hat{\theta}_{m} &= \argmax_\theta{ \ln{P(\mathbf{V} | \theta)} } \\
                 &= \argmax_\theta{\big( -{\ln{\pi^n}} - \ln{(\det{R})} - 
                      \mathbf{V}^HR^{-1}\mathbf{V} \big)} \\ 
                 &= \argmin_\theta{\big( \ln{(\det{R})} + \mathbf{V}^HR^{-1}\mathbf{V} \big)} \\
                 &\approx \argmin_\theta{\big( \mathbf{V}^H
                    (\mathbf{G}(\theta)\mathbf{G}(\theta)^H)^{-1}
                    \mathbf{V} \big)}
  \end{aligned}
\end{equation}
Bartlett's estimator simplifies this computation even further. (\textbf{TODO} explanation 
and references needed.)
\begin{equation} \label{eq:bartlett}
  \hat{\theta}_b = \argmax_\theta{\big( \mathbf{V}^H
        \mathbf{G}(\theta)\mathbf{G}(\theta)^H\mathbf{V} \big)}
\end{equation}

  \subsection{Search space} 
We now define the search space for position estimation algorithms in terms of the direction 
finder $\hat{\theta}_b$. Suppose we have a set of pulse signals $\mathcal{V}$ from multiple receiver 
sites within a particular window of time. Let $P_1, P_2, \dots P_n$ be the coordinates of the 
receivers and suppose we have a candidate position $P$ of the transmitter within the given 
time window. Let $\theta_k$ be the bearing from $P$ to $P_k$ for $1 \le k \le n$. Finally, 
we define the following function: 
\begin{equation*}
  f(\theta_k | \mathbf{V}) = 
    \mathbf{V}^H\mathbf{G}(\theta_k)\mathbf{G}(\theta_k)^H\mathbf{V} 
  \qquad \text{for $\mathbf{V} \in \mathcal{V}$. (See (\ref{eq:bartlett}).)}
\end{equation*}
where $k$ is the source site of $\mathbf{V}$, i.e., the receiver where the pulse was recorded.
We say that the log likelihood of $P$ being the true position of the target wihtin the time 
window is the following quantity: 
\begin{equation}
  F(P) = \sum_{\mathbf{V} \in \mathcal{V}}{f(\theta_k | \mathbf{V}}) 
  \qquad \text{$k$ source of $\mathbf{V}$.}
\end{equation}
The goal of position estimation is to find a point $P$ that maximizes this 
function. 

\pagebreak
\begin{figure}
  \vspace{-60pt}
  \begin{center}
    \includegraphics[scale=0.5]{figures/ll_space.png}
  \end{center}
  \caption{Distribution of $F(P)$ over the Quail Ridge reserve. 
   The "ray"-like artifacts correspond to regions where callibration data 
   is sparse.}
  \vspace{-10pt}
\end{figure}


\begin{appendices}

\section{Database definitions}
This appendix defines the various parameters in the QRAAT system in terms of the 
database schema. 

\subsection{Signal model} 
Transmitter pulses are stored in the \texttt{est} table of the QRAAT database. This section 
provides a mapping of the various parameters in the signal model to the schema. The complex 
signal vector $\mathbf{V}$ is the eigenvalue decomposition of the sampled pulse.
$$ \mathbf{V} \equiv 
  \langle 
    \texttt{ed}1\texttt{r} + i \cdot \texttt{ed}1\texttt{i}, \quad
    \texttt{ed}2\texttt{r} + i \cdot \texttt{ed}2\texttt{i}, \quad \dots \quad
    \texttt{ed}n\texttt{r} + i \cdot \texttt{ed}n\texttt{i}  
  \rangle $$
The noise covariance matrix is also stored in the \texttt{est} table. 
$$ \Sigma_{x,y} \equiv \Big[ \texttt{nc}xy\texttt{r} + 
                            i \cdot \texttt{nc}xy\texttt{i} \Big] $$
$$ t \equiv \texttt{edsp}, \quad \sigma_T^2 = \texttt{edsp} - 
                                 \texttt{trace(}\Sigma\texttt{)} $$ 

The steering vectors $\mathbf{G}(\theta)$ are stored in the \texttt{Steering\_Vectors} table.
Because these are simply derived from pulses interpolated with known GPS coordinates, we 
store their eigenvalue decomposition just like the \texttt{est} records. 
$$ \mathbf{G}(\theta) \equiv 
  \langle 
    \texttt{sv}1\texttt{r} + i \cdot \texttt{sv}1\texttt{i}, \quad
    \texttt{sv}2\texttt{r} + i \cdot \texttt{sv}2\texttt{i}, \quad \dots \quad
    \texttt{sv}n\texttt{r} + i \cdot \texttt{sv}n\texttt{i}  
  \rangle $$
where $\texttt{Steering\_Vectors.Bearing} = \theta$ and $\theta \in \{ 0, 1, 2, \dots 359 \}$.  
\end{appendices}

\end{document}
